% Template for PLoS
% Version 3.5 March 2018
%
% % % % % % % % % % % % % % % % % % % % % %
%
% -- IMPORTANT NOTE
%
% This template contains comments intended
% to minimize problems and delays during our production
% process. Please follow the template instructions
% whenever possible.
%
% % % % % % % % % % % % % % % % % % % % % % %
%
% Once your paper is accepted for publication,
% PLEASE REMOVE ALL TRACKED CHANGES in this file
% and leave only the final text of your manuscript.
% PLOS recommends the use of latexdiff to track changes during review, as this will help to maintain a clean tex file.
% Visit https://www.ctan.org/pkg/latexdiff?lang=en for info or contact us at latex@plos.org.
%
%
% There are no restrictions on package use within the LaTeX files except that
% no packages listed in the template may be deleted.
%
% Please do not include colors or graphics in the text.
%
% The manuscript LaTeX source should be contained within a single file (do not use \input, \externaldocument, or similar commands).
%
% % % % % % % % % % % % % % % % % % % % % % %
%
% -- FIGURES AND TABLES
%
% Please include tables/figure captions directly after the paragraph where they are first cited in the text.
%
% DO NOT INCLUDE GRAPHICS IN YOUR MANUSCRIPT
% - Figures should be uploaded separately from your manuscript file.
% - Figures generated using LaTeX should be extracted and removed from the PDF before submission.
% - Figures containing multiple panels/subfigures must be combined into one image file before submission.
% For figure citations, please use "Fig" instead of "Figure".
% See http://journals.plos.org/plosone/s/figures for PLOS figure guidelines.
%
% Tables should be cell-based and may not contain:
% - spacing/line breaks within cells to alter layout or alignment
% - do not nest tabular environments (no tabular environments within tabular environments)
% - no graphics or colored text (cell background color/shading OK)
% See http://journals.plos.org/plosone/s/tables for table guidelines.
%
% For tables that exceed the width of the text column, use the adjustwidth environment as illustrated in the example table in text below.
%
% % % % % % % % % % % % % % % % % % % % % % % %
%
% -- EQUATIONS, MATH SYMBOLS, SUBSCRIPTS, AND SUPERSCRIPTS
%
% IMPORTANT
% Below are a few tips to help format your equations and other special characters according to our specifications. For more tips to help reduce the possibility of formatting errors during conversion, please see our LaTeX guidelines at http://journals.plos.org/plosone/s/latex
%
% For inline equations, please be sure to include all portions of an equation in the math environment.
%
% Do not include text that is not math in the math environment.
%
% Please add line breaks to long display equations when possible in order to fit size of the column.
%
% For inline equations, please do not include punctuation (commas, etc) within the math environment unless this is part of the equation.
%
% When adding superscript or subscripts outside of brackets/braces, please group using {}.
%
% Do not use \cal for caligraphic font.  Instead, use \mathcal{}
%
% % % % % % % % % % % % % % % % % % % % % % % %
%
% Please contact latex@plos.org with any questions.
%
% % % % % % % % % % % % % % % % % % % % % % % %

\documentclass[10pt,letterpaper]{article}
\usepackage[top=0.85in,left=2.75in,footskip=0.75in]{geometry}

% amsmath and amssymb packages, useful for mathematical formulas and symbols
\usepackage{amsmath,amssymb}

% Use adjustwidth environment to exceed column width (see example table in text)
\usepackage{changepage}

% Use Unicode characters when possible
\usepackage[utf8x]{inputenc}

% textcomp package and marvosym package for additional characters
\usepackage{textcomp,marvosym}

% cite package, to clean up citations in the main text. Do not remove.
% \usepackage{cite}

% Use nameref to cite supporting information files (see Supporting Information section for more info)
\usepackage{nameref,hyperref}

% line numbers
\usepackage[right]{lineno}

% ligatures disabled
\usepackage{microtype}
\DisableLigatures[f]{encoding = *, family = * }

% color can be used to apply background shading to table cells only
\usepackage[table]{xcolor}

% array package and thick rules for tables
\usepackage{array}

% create "+" rule type for thick vertical lines
\newcolumntype{+}{!{\vrule width 2pt}}

% create \thickcline for thick horizontal lines of variable length
\newlength\savedwidth
\newcommand\thickcline[1]{%
  \noalign{\global\savedwidth\arrayrulewidth\global\arrayrulewidth 2pt}%
  \cline{#1}%
  \noalign{\vskip\arrayrulewidth}%
  \noalign{\global\arrayrulewidth\savedwidth}%
}

% \thickhline command for thick horizontal lines that span the table
\newcommand\thickhline{\noalign{\global\savedwidth\arrayrulewidth\global\arrayrulewidth 2pt}%
\hline
\noalign{\global\arrayrulewidth\savedwidth}}


% Remove comment for double spacing
%\usepackage{setspace}
%\doublespacing

% Text layout
\raggedright
\setlength{\parindent}{0.5cm}
\textwidth 5.25in
\textheight 8.75in

% Bold the 'Figure #' in the caption and separate it from the title/caption with a period
% Captions will be left justified
\usepackage[aboveskip=1pt,labelfont=bf,labelsep=period,justification=raggedright,singlelinecheck=off]{caption}
\renewcommand{\figurename}{Fig}

% Use the PLoS provided BiBTeX style
% \bibliographystyle{plos2015}

% Remove brackets from numbering in List of References
\makeatletter
\renewcommand{\@biblabel}[1]{\quad#1.}
\makeatother



% Header and Footer with logo
\usepackage{lastpage,fancyhdr,graphicx}
\usepackage{epstopdf}
%\pagestyle{myheadings}
\pagestyle{fancy}
\fancyhf{}
%\setlength{\headheight}{27.023pt}
%\lhead{\includegraphics[width=2.0in]{PLOS-submission.eps}}
\rfoot{\thepage/\pageref{LastPage}}
\renewcommand{\headrulewidth}{0pt}
\renewcommand{\footrule}{\hrule height 2pt \vspace{2mm}}
\fancyheadoffset[L]{2.25in}
\fancyfootoffset[L]{2.25in}
\lfoot{\today}

%% Include all macros below

\newcommand{\lorem}{{\bf LOREM}}
\newcommand{\ipsum}{{\bf IPSUM}}





\usepackage{forarray}
\usepackage{xstring}
\newcommand{\getIndex}[2]{
  \ForEach{,}{\IfEq{#1}{\thislevelitem}{\number\thislevelcount\ExitForEach}{}}{#2}
}

\setcounter{secnumdepth}{0}

\newcommand{\getAff}[1]{
  \getIndex{#1}{}
}

\providecommand{\tightlist}{%
  \setlength{\itemsep}{0pt}\setlength{\parskip}{0pt}}

\begin{document}
\vspace*{0.2in}

% Title must be 250 characters or less.
\begin{flushleft}
{\Large
\textbf\newline{CEEDS Weather Dashboard} % Please use "sentence case" for title and headings (capitalize only the first word in a title (or heading), the first word in a subtitle (or subheading), and any proper nouns).
}
\newline
% Insert author names, affiliations and corresponding author email (do not include titles, positions, or degrees).
\\
Julia Lee\textsuperscript{},
Marta García\textsuperscript{},
Mirella Hernandez\textsuperscript{}\\
\bigskip
\bigskip
\end{flushleft}
% Please keep the abstract below 300 words
\section*{Abstract}
We were tasked with finding a way to better showcase weather data from
the MacLeish Field station, as well with incorporating interactive
graphics and allowing users to download and filter the data. Due to the
capabilities of Shiny, we decided to use the Shiny and Shiny Dashboard
packages in R to build an interactive dashboard. We also wrote a package
to get the data from the MacLeish server.

% Please keep the Author Summary between 150 and 200 words
% Use first person. PLOS ONE authors please skip this step.
% Author Summary not valid for PLOS ONE submissions.

\linenumbers

% Use "Eq" instead of "Equation" for equation citations.
\section{Introduction}\label{introduction}

The Center for the Environment, Ecological Design, and Sustainability
(CEEDS) is located in Wright Hall building of Smith College. The goal of
Ceeds is to train and allow students to consider sustainability issues
across various disciplinary areas to integrate that multi-disciplinary
knowledge in support of environmental decisions and action. Macleish
Field Station is one of the many opportunities Ceeds provides to the
Smith community to pursue environmental research, outdoor recreation,
and low impact recreation.

The MacLeish Field Station is located in a 260 acre of forest farmland
near Whately, Massachusetts, and its weather collection sites are at the
end of Poplar Hill Road. There are two weather site locations in the
field, one called WhatelyMet Tower and the other one called OrchardMet.
The WhatelyMet tower is mounted at the top of a 25.3m tall tower, and it
is 250.8m above sea level. It was initially created to collect air
pollution data. In order to collect more efficient weather data,
OrchardMet weather station was created. The OrchardMet weather station
is 10m tall located in a forest clean area close to ground level.
However, the trees do not allow OrchardMet weather station to collect
data efficiently as hoped.

As part of the SDS 410 Capstone class, our group of 3 were given
Macleish weather data from the Macleish package {[}1{]} that is
collected through the Smith server. There is a current Macleish weather
dashboard that collects a variety of weather data: date, wind speed,
temperature, wind direction, real humidity, pressure, solar radiation,
and rainfall. The server has been collecting data since January 2012,
and it is updated every 10 minute increments. Thus, there is over
377,016 entries. The data loggers run on solar power and stores the data
so even if the server is down the data is saved. The current MacLeish
weather dashboard\footnote{current Macleish weather dashboard:
  http://macleish.smith.edu/index.html} contains a lot of useful
information, and the data is organized into readable tabs. Because the
data is trapped in individual ``data silos'', it is not readily
accessible. Therefore, the goal for the given project is to recreate the
current Macleish weather dashboard to illuminate the data and make it
more user friendly, it download accessible, and it visually interactive
as proposed by our client, Paul Wetzel. Our client Paul Wetzel is the
field station manager of MacLeish Field Station, and he is the main user
of the current website. As discussed and mentioned earlier about our
clients needs, we chose to make the new weather dashboard using the
Shiny package {[}2{]}. We choose to use Shiny, since it would allow us
to make the download data opportunity catered to the audience's needs,
and it would help us make the graphs interactive and user friendly. We
believe our focus would supply to the current target audiences, Paul
Wetzel, other Smith students, Smith faculty/researchers, and Northampton
Department of Public Works.

\begin{figure}
\includegraphics[width=350px]{current} \caption{Example of a what the current Macleish weather dashboard looks like}\label{fig:unnamed-chunk-1}
\end{figure}

\section{Methods}\label{methods}

\subsection{Shiny app:}\label{shiny-app}

To solve our problem, we had many options on how to create a better way
of displaying the weather data. We first discussed the possibility of
making a new website in HTML but we ultimately decided to build a Shiny
App using the Shiny packages in R. Shiny is a package that allows us
``to build interactive web applications (apps) straight from R.'' Shiny
allows the user to be able better communicate data with interactive
charts, visualizations, text and tables. We then decided to use Shiny
Dashboard which is a package that allows us to create an interactive
dashboard using Shiny. Because our problem was to find a way to display
interactive data visualizations, the Shiny package seemed to be the
better option because that is what it was built to do. Shiny also allows
us to make normally static plots like ggplots interactive and take in
user input so we don't need to use HTML widgets.

Shiny is based on a reactive programming model which means that it takes
in input given by the user of the app and acts accordingly. Shiny apps
have two components: the user interface (UI) and the server. You can
think of the UI as what the user of your app will see when they open
your app. The UI component converts your code in R and generates it into
a web document in HTML. UI contains the instruction about what you want
the user to see and the layout of your app. our UI is where we define
the dashboard page. And inside this dashboard page we define our
elements like the sidebar menu, tabs, the dashboard body (Tab items,
boxes, text). The server is the R code that tells your shiny server what
to do when the user does certain things in the app. The server is a
function where we define the output and tell the host of our app what to
do with the user input. Server function is where we render the charts,
visualizations, and tables. We then knit together the Server function
and UI using the ShinyApp (UI, server) function in the Shiny package.

So we built a Shiny dashboard. Our dashboard has a sidebar menu that has
four tab items. One tab item we built was the current weather tab. The
current weather tab gives the user the the current 10 minute weather for
WhatelyMet and OrchardMet. Because we wanted to have our data be shown
as a text object and not a table, our code for this tab was largely done
using HTML. We used a bit of HTML to change the sizing of the text and
the alignment of the rows. To make the teal boxes, we used the
shinydashboard Plus package {[}3{]}, allowed us to make a clear, concise
first page of our Shiny app. Another tab on our dashboard is the about
tab where we explain our project and give information about the weather
stations.

\begin{figure}
\includegraphics[width=350px]{Weatherdashboard} \caption{Example of a what our sidebar menu and dashboard looks like}\label{fig:unnamed-chunk-2}
\end{figure}

Another tab that we built was the historic data tab. We made a way for
the user to switch between daily data for OrchardMet and WhatelyMet in
the historic data tab, this means the graphs shown are reactive and are
based on the data set that was selected by the user. We did by making a
reactive function inside the server that allows people to switch between
data sets, this function we call datasetInput. We made graphs mapping 3
(Wind speed, precipitation, temperature) variables over time. To do
build these graphs we used the highcharter package {[}4{]} that allows
the user to filter the data based on a certain time period and high
charter also has a feature that enables users to download the data used
to make the graphic.

\begin{figure}
\includegraphics[width=350px]{highchart} \caption{Example of a Highcharter graph that shows temperature}\label{fig:unnamed-chunk-3}
\end{figure}

We also made a visualization where the user can make a custom
correlation graph by picking the variables that they want. We made this
correlation scatterplot by using ggplot and ggplotly to make the graphic
more interactive. We also made a windrose graph to show wind speed and
direction using ggplot. A wind rose is a polar chart which shows the
user how wind speed and wind direction are distributed at a over a
specific period of time. We also created a way to allow the user to
choose a variable and get summary statistics on that variable. In this
tab, we made the graphs be in a fluid page, this allows for the size of
the boxes and visuals to change based on the size of the window. We then
put all of our content for the tab in separate boxes that are
collapsible. We did this to minimize users' scrolling.

\begin{figure}
\includegraphics[width=350px]{windrose} \caption{A wind rose graph}\label{fig:unnamed-chunk-4}
\end{figure}

The last tab we made is the raw data tab. This tab allows the user to
filter the data based on columns and the user can also choose between
daily data for OrchardMet and WhatelyMet. This tab also allows the user
to search the data table using the DT package.

\subsection{Ceeds package:}\label{ceeds-package}

The data from the two weather stations is currently being saved on the
Macleish Server hosted at Smith college. We used the Macleish package to
fetch the two data tables from the server. However this involved a lot
of code that would have to be repeated anytime we wanted to run the app.
So we decided to write an R package called ``ceeds'' to more efficiently
do these tasks. This package was also built Help other students to work
on the MacLeish weather data. We wrote a function that uses the Macleish
package and fetches 2 data sets (Whately, Orchard) that is updated every
ten minutes. We also created a function,get\_daily(), inside the package
that takes a data set and gives the user the data grouped by date. Our
data from the server comes in 10 minute increments but we might want to
group by date. So we used the Lubridate package that was very helpful.
It allowed us to Parsed timestamps into dates (turning variables into
dates that r can recognize as such). This allowed us to group our data
by date. {[}5{]}

\section{Issues}\label{issues}

Throughout this whole process, our group has faced a variety of issues
and consequently, temporary limitations. At the beginning of our
process, we first had to learn to use the Shiny package. There was a
learning curve in learning to build a web app and we initially struggled
with building the first parts of the app and figuring out how to
integrate the R knowledge we had with this new package. We took the
DataCamp course to learn the basics of Shiny and from there, dove
straight into our new data, once we had access to the data. Since we
were reliant on the MacLeish package, which relied on the MacLeish
server to receive the data, whenever the MacLeish server was down, we
struggled with working with the data we needed to move forth with the
project.

Also at the start of this project, we met with our client to gather more
information of what he wanted and what we felt we could deliver knowing
our skill set in R and Shiny. Since Wetzel was not familiar with the
capabilities of R, it was up to us to fill him in with what we thought
was possible. These plans included reducing the amount of tabs, changing
all the graphs to being clearly interactive, and making the web app
overall more user-friendly and comprehensive. We did not create a
printed mockup until much later in the process so it was difficult to
communicate our vision to him and for him to understand the potential of
R and Shiny. However, in our biweekly meetings with him, we filled him
in with whatever progress we were making and once we started creating
visualizations using highcharter, he was able to better understand the
kind of graphs we could create and integrate into the new web app.

In the similar vein as to how Wetzel did not know much about R, our
group did not know much about weather data and the importance of all the
variables. During our meeting Wetzel taught us the importance of certain
variables but also told us that we should focus on the precipitation,
temperature and wind variables if we needed to narrow our scope. In
teaching us about the other variables, such as soil temperature and
server temperature and this helped us figure out which variables were
most important to the user. For example, the server room temperature
would be most useful to the weather station manager (Wetzel) because
that temperature would allow him to see when the room was overheating or
too cold, and not very functional for a user looking to download weather
data for a school project. Through Wetzel, we also learned that
temperature data is only accurate up to the thousandth place which made
us modify our app to reflect that accuracy.

Arguably our largest and most time-consuming issue, making tabs work in
Shiny took a while. Since our project is so visual, and tabs played a
major part in our functioning dashboard, this set us back and we were
unable to move forward with the HTML \& CSS coding until we could solve
this problem. We had already programmed in some graphs but with tabs not
working, they would all appear on the same page and when the user tried
to click to different tabs, they would not switch. However, with some
help from our professor Ben Baumer, we were able to get tabs to work.
Through this, we learned the importance of correct placement of commas
and parentheses in Shiny since as we discovered later, that had been our
problem the whole time. Once tabs worked, we were able to put all the
appropriate graphs in each tab and start working on the user interface
that we coded mostly with HTML \& CSS.

A smaller issue that we had was making graphs reactive. Since we were so
familiar with using ggplot package through tidyverse, we wanted to have
all our data visualizations be programmed using ggplot. However, we had
a lot of issues with making these visualizations reactive with Shiny,
which posed a problem of how a user would then interact with these
graphs. After trying for a short while, we ultimately moved onto a
different package called highcharter, which made the graphs interactive
and also allowed an element of downloading data that we expanded on
later with a tab we included.

\section{Moving Forward}\label{moving-forward}

There are many things that can looked at in future to expand this
project. A future endeavor of our project is that we have not yet
surveyed our target audience about whether our app is more usable than
the current dashboard. Audience feedback would help us moving forward in
reach our end goal of creating a user friendly weather dashboard. We
have already tested our dashboard with a group of peers, however we
would have liked to test on a wider audience.

After talking to our client, we found out that sometimes the weather
equipment breaks and leads to wrong data but we don't necessarily want
to filter out possible outliers of the data assuming that they mean
equipment malfunctions so we might want the users of dashboard to make
that decision to not mislead them. We might also want to have more
variables in our data set like soil temperature which is used to study
climate change. We would also liked to implement some sort of special
setting that would have extra tabs for our client, the weather manager.
These extra tabs would include data that is important to him, such as
soil temperature and the bunker room temperature, both tabs we felt were
not important to our general audience of students and others looking to
check the weather or download a large amount of data.

Since the data is recorded by weather instruments that are sometimes
faulty, the data that is sent to servers is not always accurate. Due to
this, Wetzel often goes in and manually changes the data. We would have
liked to either include a small disclaimer that would tell people
downloading or analyzing the data about this since the data used in the
dashboard is server data, not the corrected data. Besides that, we also
wished to include the corrected data in our app, in order to have both
datasets on the dashboard.

Right now our dashboard has very few features and could have more places
for user interaction. For example, Our wind rose graphic is not
interactive which could improve this project. Another feature we might
want to have in the future isa way to allow users to identify and filter
out outliers.

\section*{References}\label{references}
\addcontentsline{toc}{section}{References}

{[}6{]}

\hypertarget{refs}{}
\hypertarget{ref-macleish}{}
1. Baumer BS, Goueth R, Li W, Zhang W, Horton N. Macleish: Retrieve data
from macleish field station {[}Internet{]}. 2018. Available:
\url{https://CRAN.R-project.org/package=macleish}

\hypertarget{ref-shiny}{}
2. Chang W, Cheng J, Allaire J, Xie Y, McPherson J. Shiny: Web
application framework for r {[}Internet{]}. 2018. Available:
\url{https://CRAN.R-project.org/package=shiny}

\hypertarget{ref-dashboardplus}{}
3. Granjon D. ShinydashboardPlus: Add more 'adminlte2' components to
'shinydashboard' {[}Internet{]}. 2019. Available:
\url{https://CRAN.R-project.org/package=shinydashboardPlus}

\hypertarget{ref-highcharter}{}
4. Kunst J. Highcharter: A wrapper for the 'highcharts' library
{[}Internet{]}. 2019. Available:
\url{https://CRAN.R-project.org/package=highcharter}

\hypertarget{ref-ceeds}{}
5. Baumer BS, García M, Hernandez M, Lee J. Ceeds. 2019.

\hypertarget{ref-ceeds_repo}{}
6. Baumer BS, García M, Hernandez M, Lee J. Ceeds github repository.
\url{https://github.com/beanumber/ceeds}; 2019.

\nolinenumbers


\end{document}

